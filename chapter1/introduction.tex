
\chapter{Introduction} \label{introduction}

\section{Overview of Virtualization}
Hardware resource abstraction and management have always been a major aim and challenge in
electronic systems. With the advent of computers these tasks fell upon a huge piece of software
called the Operating System. It became the responsibility of the OS to multiplex the resources
among the different processes running in a system. An obvious illustration is that of CPU
multiplexing, where the CPU is scheduled in time slices among various processes according to
priority. The physical memory in a system as well is a resource. Memory also needs to be
abstracted and suitable interfaces are to be provided for easy and efficient use. Traditionally
this has been achieved through the concept of virtual memory coupled with paging or segmentation.
\\
With advancements in technology the capabilities of physical resources increased and cost
decreased. Processors and memory became faster, smaller chips with more capacity became available.
But these advancements didn't percolate naturally into better utilization of these resources.
Often these resources were underutilized. In order to get better utility from these resources
various options of sharing these resources at a larger granularity was explored. Thus came into
existence the concept of virtualization.\\
Virtualization increases the granularity of abstraction from individual resources to abstracting
the entire set of hardware as a single unit or in other words virtualization abstracts the entire
computer system. With this we could have more than one machine running on top of the existing
computing hardware. These machines came to be called \textit{Virtual Machines} or VM. In other
words with the increase in granularity of abstraction the unit of allocation to the abstraction
increased from a process to an entire OS.\\
The ringmaster who runs the show in a Virtual Machine is still the humongous piece of code called
the OS. But traditionally OSes were designed to have ownership and control of the underlying
hardware. Virtualization now created a separation between the hardware and the OS.They were now
relegated to the status of a \textit{guest OS}. A single OS was no longer the sole owner and
controller of the resources. The entire hardware had to be abstracted, interfaced and multiplexed
among different OSes manning different VMs analogous to the way in which an OS enabled
multiplexing of resources among different processes in a native system. This role was now taken up
by a new, but no less hideous, software layer called the \textit{Virtual Machine Monitor} (VMM) or
the \textit{Hypervisor}. That is now we have the hypervisor sitting on top of the hardware 
directly and above it we have different VMs.\\
The introduction of a new orchestrator, the Hypervisor, and relegation of a guest OS to a lesser
privilege brought about new challenges. The guest OS in a VM had no longer complete control of the
underlying resources to arbitrate efficiently among the processes running in it. But to the
process local to a VM the guest OS was still the ringmaster who ran the show. Hence it became of
paramount importance that the introduction of a software layer, the hypervisor, between the guest
OS and the resources didn't break the equivalence view  i.e. a process running in a VM should see
no or little difference in running on a native vs. virtualized system. At the same time we also
had to ensure that every VM stayed within its allocated bounds and guest OS operations had enough
efficiency. These requirements of hypervisor design are formally stated in 
\cite{popek1974formal}.The hypervisor can choose from among the various strategies like
instruction interpretation, trap \& emulate, binary translation, para-virtualization, hardware
assisted virtualization to provide the aforementioned requirements.\\
Once such a hypervisor is available efficient resource utilization and management comes next in
order to meet the original design principles of virtualization. Memory like other
resources will also be apportioned to the different VMs. But it is not necessary that all  of the
memory allocated to a VM will be in 100\% use all the time. This gives us an opportunity to
reallocate this unused memory to other VMs in need. But its sharing and management is not as
simple as that of a flexible resource like CPU and nor are the effects of differences in the
amounts of memory available that easily visible \citep{hwang2013mortar}. The objective of this
seminar is to get an understanding of the intricacies involved in this. 

\section{Scope of the Seminar} \label{scope}
The objective of this seminar is not to delve deep into the implementation of virtualization as a
whole but rather to concentrate on understanding how memory virtualization is achieved, the pros and cons of different techniques of memory virtualization, optimizations to it, challenges of memory management in virtualized systems, memory reallocation mechanisms, and some of the existing memory management strategies.     

%\begin{itemize}
%\item Unnumbered and Numbered Lists
%\item Equations
%\item Defining short macros for frequently used symbols
%\item Bibliography
%\item Figures
%\item Tables
%\end{itemize}

%The normal procedure for compiling a \LaTeX\ document that contains
%bibliographic entries is to follow the following steps
%\begin{enumerate}
%\item \verb|pdflatex mainrep|
%\item \verb|bibtex mainrep|
%\item \verb|pdflatex mainrep|
%\item \verb|pdflatex mainrep|
%\end{enumerate}
%In the above example \verb|mainrep| is the main \LaTeX\ file.




%This is the first chapter, which resides in a directory (folder)
%intro. Each chapter can contain \verb|section|, \verb|subsection|
%and so on.




%Equations should be set in a separate mode.  For details on getting
%various types of aligned equations, consult the \AmS-\LaTeX\
%documentation \verb|amsldoc.pdf|. Simple equations are set as
%\begin{equation}
%\label{eq:sinx}
%\int \mathrm{d}x \; \cos x =  \sin x
%\end{equation}
%Equation~\eqref{eq:sinx} is the integral of the cosine
%function. Mathematical symbols must always be put inside \verb|$$|,
%when they appear outside a math environment (such as \verb|equation|,
%\verb|align|, \verb|gather|, etc).  The symbol ``ex'' must be written as
%$x$ and not as x.  

%Another commonly used construct for equations is the \verb|align|
%environment to align several equations along a vertical line. It is
%usually the $=$ sign across which the alignment is done.  The
%point of alignment for each equation is specified using the ampersand symbol 
%\begin{align}
%a &= b  \\
%a + e + f + g & = m + n + z \\
%x + 2 & = x^{3} + 3 x^{2} + 2 x + 5
%\end{align}

%\subsection{Commonly used Symbols}
%For mathematical symbols it is very convenient to define frequently
%used symbols as a short macro. For example if you are to be using the
%symbol $\eta_{\mathrm{s}}$ frequently it is convenient to define it in
%as:\\
%\verb|\newcommand{\etas}{\ensuremath{\eta_{\mathrm{s}}}}| \\
%n the preamble and to simply refer it to in the text as \etas\ or in
%a mathematical equation as $\etas = \eta \, ( 1 + \phi)$.

%%% Local Variables: 
%%% mode: latex
%%% TeX-master: "../mainrep"
%%% End: 
