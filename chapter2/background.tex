\chapter{Background}

\section{Memory Management in Native Systems}
Memory management in non-virtualized native multiprogramming system is achieved using Virtual
Memory. In multiprogrammed systems several programs are resident in memory at the same time.
The memory management policy in such a system deals with protecting the memory of one program from 
another, loading a program into available space in main memory, de/allocating memory dynamically
from/to programs.\\
While programs are complied and linked with addresses starting at $0$ and CPU uses 
these addresses to access the binary, it isn't necessary (and is the not the case that) that a
program will get physical memory with the same addresses or it is not even guaranteed that there
will be enough space in the physical memory to load the entire binary of the program. Most of the
times only the immediately needed part of the program is loaded onto the available physical memory.
Hence a program is given the illusion that there is enough physical memory available to 
Hence the addresses generated by CPU needs to be translated into the corresponding physical
addresses. The address generated by CPU is called \textit{Virtual Address} or VA.
