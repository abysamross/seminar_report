\chapter{Extra info \& notes}

1. change to openright in iitb.cls file 
2. set cleardoublepage after listofmainrep.tex



As mentioned in the introduction section the arrival of a new layer of software abstraction above
the hardware relegated the native OS to the role of a guest OS. Especially on architectures like the
x86 this posed a serious challenge. x86 architecture has this notion of different CPU privilege
levels to provide protection. They are represented as \textit{protection rings}. There are 4 rings
and each ring correspond to a privilege level, 0 being the highest privilege level and 3 the lowest.
The OS in a normal setting runs on \textit{Ring 0} and user applications on \textit{Ring 3}.
\textit{Rings 1 \& 2} are not used in general but was originally intended by
Intel\textsuperscript{TM} for housing device drivers. They have more permissions than \textit{Ring
3} but cannot access privileged instructions and any attempt to access such an instruction results
in a general protection fault or \textit{GFP} \citep{wiki1}. This gives the OS complete control over
all resources.\\
The introduction of a hypervisor made it necessary to host it on \textit{Ring 0} and the guest OS
had to be moved down to a lower privilege level. Now we would still want the guest OS to behave as
if it is operating in \textit{Ring 0}. Normal instructions executed by the guest OS can go ahead
without any hindrance analogous to the way a user mode instruction executes in a non-virtualized
setup. But whenever the guest OS is trying to execute a privileged instruction the the hypervisor
should somehow translate/emulate the effects of those privileged instructions as the guest OS longer
has enough privileges to execute them on its own. This in turn means that there should be ways for
the hypervisor to know when a guest OS is trying to execute a privileged instruction. Normally this
is ensured because as the guest OS now has lower privileges trying to execute the privileged
instruction causes a general protection fault.\\
But there is a subtle issue here. Not all privileged instructions behave the same way when executed
from a non-privileged mode. Certain privileged instructions when executed from a non-privileged mode
does not cause a general protection fault and instead is replaced with a \textit{noop}. Hence the
hypervisor cannot translate/emulate these instructions for the guest OS. This was resolved by moving
the guest OS to \textit{Ring 1} where all privileged instructions generate a trap.
